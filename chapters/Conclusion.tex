%================================================================
\chapter{Conclusion and perspectives}\label{chap:conclusion}
%================================================================

The associated production of $\Upsilon(nS) + $ D$^{*\pm}$ analysis has been presented. The aim was to provide a cross section measurement of the associated production and to determine the $\sigma_{eff}$ using data from pp collisions from LHC at $\sqrt{s} = 13$ TeV, recorded by CMS during Run 2 between 2015 and 2018. The total integrated luminosity of the data samples amounts to 123 fb$^{-1}$.

The measured fiducial cross sections for each of the $\Upsilon$ states, in the fiducial region defined by the Table \ref{tab:fiducialvol}, are
\begin{equation}
\begin{split}
    \sigma_{Y(1S)D^{*\pm}} &= 474^{+44}_{-68} \text{(stat)} \pm 30 \text{(syst)} \; \text{pb},\\
    \sigma_{Y(2S)D^{*\pm}} &= 230^{+23}_{-34} \text{(stat)} \pm 19 \text{(syst)} \; \text{pb},\\
    \sigma_{Y(3S)D^{*\pm}} &= 152^{+16}_{-22} \text{(stat)} \pm 17 \text{(syst)} \; \text{pb}.
\end{split} 
\end{equation}
It is worth noting that there are further developments needed, mainly in the efficiencies calculations, which was done from a MC with restricted statistics, and a more deep determination of the systematic uncertainties.

For the effective cross section, the fiducial region had to be restricted to the one from previous measurements of $\Upsilon(nS)$ and D$^{*\pm}$ cross sections. This resulted in a bad agreement between the numbers from the $\Upsilon(2S)$ and $\Upsilon(3S)$. For this reason, this number was calculated only for $\Upsilon(1S)$, which showed good agreement between the samples. By assuming associated production is free from SPS contribution, the minimum value for the effective cross section is
\begin{equation}
\sigma_{eff} > 8.1^{+1.0}_{-1.2}\text{(stat)} \; \text{mb}.
\end{equation}
In this measurement, a way to improve the result was pointed out in Sec. \ref{sec:discussion}, which has the advantage to use the full kinematical region proposed by this analysis and to have lower systematic uncertainties sources. Also, a new MC, with higher statistics is being processed, so the efficiencies will be better determined.

Regarding the CMS RPC project, it has been presented the contributions given in the maintenance, operation and R\&D. There are many challenges for these detectors, as they have to keep the excellent timing, stability and robustness, even in the high radiation environment of the HL-LHC. Finally, because of the regulations imposed to the high GWP gases, the search of a low GWP gases to substitute the freon and SF$_6$ are of most importance for the future of this gaseous detector technology.