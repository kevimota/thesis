\chapter*{Introduction}



This work is structured as follows. Chapter \ref{chap:theo} presents a theoretical overview of the Standard Model and the Multiple parton scattering processes, focusing on the main channel of analysis $\Upsilon + $D$^*$. General aspects of the collider and the experimental apparatus are discussed in Chapter \ref{chap:experiment}. Chapter \ref{chap:rpc} reviews the gaseous detectors with emphasis on the resistive plate chambers in CMS and the contributions given by the author to the CMS RPC project. Finally, in Chapter \ref{chap:conclusion} a summary and perspectives for future development are presented

Throughout this document, the following conventions are adopted:
\begin{itemize}
    \item Lowercase latin indexes, e.g. i, j and k, vary on the three spatial coordinates, generally as 1, 2 e 3 or x, y e z;
    \item Lowercase greek indexes, e.g. $\mu$, $\nu$ and $\lambda$, vary on the four spacetime coordinates;
    \item The Einstein summation convention is used.
    \item The speed of light, Planck constant and the electric permittivity are assumed to be equal to one.
    \item Both particle and antiparticle states are taken for the cross section calculation.
\end{itemize}