\chapter*{Introduction}

The Standard Model (SM) of Particles is currently the best theoretical framework for describing the behavior of the subatomic particles and three of the four fundamental interactions. It was not only able to predict the existence of many particles, such as the existence of the W and Z bosons, gluon, top and bottom quarks, but also to their properties. The last big discovery from the SM was the Higgs boson, in 2012 by both CMS and ATLAS experiments in proton-proton (pp) collisions at LHC. Nowadays, with the huge amount of data recorded from LHC collisions, the SM measurements have reached high precision and there is great expectation for discovery of physics beyond the SM leading to discovery of new phenomena and to broadening our understanding of the universe.

During pp collisions, more than often, more than one pair of partons (quark or gluon) can interact in a single collision, this is known as Multiple Parton Scattering (MPS) and is one of the predictions of the Standard Model. While it is within the current physics framework, the current models still need more refinement. Also, the analysis of the experimental data can be difficult to interpret, as the multiple interactions lead to additional activity in the detector. The MPS can be a important background, as it can mimic signals of new physics and it is expected to have higher contribution at higher energies.

The first observation of MPS was done by the AFS Collaboration by analyzing events with four-jets. It was shown that the number of events could only be explained by the MPS. Later, CDF and D0 gave important contributions and at Tevatron also exploring data with multiple jets and data with more than one quarkonia. Today, ATLAS, CMS and LHCb gave important contributions, with much more precision than ever, and being able to explore much rarer signatures, such as the triple J/$\psi$ \cite{CMS:2021qsn}. 

This thesis aims to contribute to the MPS by doing a measurement of the associated production of $\Upsilon(nS) +$ D$^{*\pm}$. This can be compared to a measurement done by LHCb for $\Upsilon(1S) + $ D$^0$ and $\Upsilon(1S) + $ D$^+$. Previous measurements of MPS show that there are differences in the measurement from the central rapidity region to the frontal one. CMS can take profit of the high recorded luminosity of 123 fb$^{-1}$ recorded during Run 2 at center-of-mass energy of 13 TeV.

In addition to the physics analysis, this document also describe the contributions given by the author to the CMS RPC project. Work was performed for the maintenance, operation and commissioning of the current subdetector as well as the upgrade, in the R\&D of the new iRPCs. 

This work is structured as follows. Chapter \ref{chap:theo} presents a theoretical overview of the Standard Model and the Multiple parton scattering processes, focusing on the main channel of analysis $\Upsilon(nS) + $D$^{*\pm}$. General aspects of the collider and the experimental apparatus are discussed in Chapter \ref{chap:experiment}. Chapter \ref{chap:rpc} reviews the gaseous detectors with emphasis on the resistive plate chambers in CMS and the contributions given by the author to the CMS RPC project. Finally, in Chapter \ref{chap:conclusion} a summary and perspectives for future development are presented.

Throughout this document, the following conventions are adopted:
\begin{itemize}
    \item Lowercase latin indexes, e.g. i, j and k, vary on the three spatial coordinates, generally as 1, 2 e 3 or x, y e z;
    \item Lowercase greek indexes, e.g. $\mu$, $\nu$ and $\lambda$, vary on the four spacetime coordinates;
    \item The Einstein summation convention is used.
    \item The speed of light ($c$), the reduced Planck constant ($\hslash = \frac{h}{2\pi}$) and the vacuum electric permittivity ($\epsilon_0$) are equal to one.
    \item Both particle and antiparticle states are considered for the cross section calculation.
\end{itemize}