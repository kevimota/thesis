\chapter{Physics Analysis}
%================================================================

This chapter will discuss about the cross section measurement of the associated production of $\Upsilon$ and $D^{*\pm}$ using the full CMS Run 2 data with 13 TeV center-of-mass energy. \textcolor{red}{falar mais}

\section{Datasets and Simulation}
\subsection{Data Samples}

The data samples were recorded by the CMS detector during the LHC Run 2 in 2016-2018 at a center-of-mass energy of 13 TeV and are composed only by events certified as good for physics analysis. The Table \ref{tab:datasamples} displays all the data samples used. The total recorded luminosity of all data samples is 137.61 fb$^{-1}$.

\begin{table}[!htbp]{15cm}
  \caption{Data samples considered in this work}\label{tab:datasamples}
  \begin{tabular}{ c }
    Dataset \\ 
    \hline
    /MuOnia/Run2016B-21Feb2020\_ver1\_UL2016\_HIPM-v1/AOD \\
    /MuOnia/Run2016B-21Feb2020\_ver2\_UL2016\_HIPM-v1/AOD \\
    /MuOnia/Run2016C-21Feb2020\_UL2016\_HIPM-v1/AOD \\
    /MuOnia/Run2016D-21Feb2020\_UL2016\_HIPM-v1/AOD \\
    /MuOnia/Run2016E-21Feb2020\_UL2016\_HIPM-v1/AOD \\
    /MuOnia/Run2016F-21Feb2020\_UL2016\_HIPM-v1/AOD \\
    /MuOnia/Run2016G-21Feb2020\_UL2016-v1/AOD \\
    /MuOnia/Run2016H-21Feb2020\_UL2016-v1/AOD \\
    \hline
    /MuOnia/Run2017B-09Aug2019\_UL2017-v1/AOD \\
    /MuOnia/Run2017C-09Aug2019\_UL2017-v1/AOD \\
    /MuOnia/Run2017D-09Aug2019\_UL2017-v1/AOD \\
    /MuOnia/Run2017E-09Aug2019\_UL2017-v1/AOD \\
    /MuOnia/Run2017F-09Aug2019\_UL2017-v1/AOD \\
    \hline
    /MuOnia/Run2018A-12Nov2019\_UL2018-v1/AOD \\
    /MuOnia/Run2018B-12Nov2019\_UL2018-v1/AOD \\
    /MuOnia/Run2018C-12Nov2019\_UL2018-v1/AOD \\
    /MuOnia/Run2018D-12Nov2019\_UL2018-v1/AOD \\
  \end{tabular}
  \legend{Data Samples used in the analysis corresponding to the full CMS Run 2 data taking.}
\end{table}

\subsection{Simulation Samples}

The simulation samples are done via Monte Carlo (MC) simulation, where the MC method is used in programs to model the physics processes. The starting point of the simulation is the event generation, where the events following a set of physics processes of interest are created by the MC generator.

First, the parton level matrix element of the process is calculated perturbatively up to a fixed order. After, the parton showering is done to take into account the higher order effects, such as initial and final state radiation (ISR and FSR). This is followed by the hadronization, where the quarks and gluons form the quarks. Finally, the simulation should also handle the particles decay.

The detector simulation follows the event generation. Here, the interaction of the generated particles with the detector is simulated. In CMS, the detector response is simulated using \textsc{Geant4} \cite{GEANT4:2002zbu}. The simulated detector response pass through the same processing chain as the real data forming the final simulated sample.

The MC samples were used in this analysis to simulate the two components of the signal, the SPS and DPS. The only background was not simulated, since it is only combinatorial

The DPS was created using the \textsc{Pythia8} \cite{Sjostrand:2014zea} as event generator, parton showering and for hadronization. For the decays of the heavy flavour hadrons, \textsc{EvtGen} \cite{Lange:2001uf} was used. The SPS sample is similar to the DPS, with the addition of the \textsc{HELAC-Onia} \cite{Shao:2012iz, Shao:2015vga}, which was used for event generation.

\textcolor{red}{colocar imagens da simulação}

\section{\texorpdfstring{$\Upsilon$ + D$^{*\pm}$}{Y+D*} Reconstruction}

The $\Upsilon$ + D$^*$ reconstruction is done from the reconstructed tracks in an event. As shown in Fig. \ref{fig:drawing_event}, there are five different particles in the final state an event - three coming from D$^*$ and two from $\Upsilon$. The D$^*$ have a low combinatorial background when compared with other open charm mesons because of it very characteristic signature - very small difference of mass between D$^*$ and D$^0$ restricting the phase space of the slow pion. Also, the kaon and pion can be identified even without a particle identification detector, since it should have a opposite charge to the slow pion, removing the ambiguity of the D$^0$ reconstruction.

\begin{figure}[!htm]{15cm}
\caption{Drawing of an event of associated production of $\Upsilon$ and D$^{*+}$}%
\label{fig:drawing_event}
\fbox{
\begin{tikzpicture}
  \coordinate (PV) at ( 0,0);
  \coordinate (SV) at ( 2.0,1.8);
  \coordinate (BL) at (-3,0);
  \coordinate (BR) at ( 4,0);

  \draw[beamcol,dashed,name path=beam]
    (BL) -- (BR);
  \draw[SVcol,dashed]
    (PV) -- (SV)
    node[midway,above=0.1,left=0.1] {$D^0$};

  \draw[->,mygreen,line width=1]
    (SV) to[bend right] (1.0,3.0) node[above right] {$\pi^+$};
  \draw[->,mygreen,line width=1]
    (SV) to[bend left] (3.0,3.0) node[above left] {$K^-$};

  \draw[->,red,line width=1]
    (PV) to[bend right=20] (-2.0,1.0) node[below right] {$\mu^+$};
  \draw[->,red,line width=1]
    (PV) to[bend left=20] (-1.0,2.0) node[below left] {$\mu^-$};


  \draw[->,SVcol,line width=1]
    (PV) arc[start angle=150, end angle=0, radius=0.5] node[below right] {$\pi_s^+$};
  
  \fill[beamcol]
    (PV) ellipse (9pt and 4.5pt)
    node[beamcol!95!black,below left] {PV $\color{red}{\Upsilon} \color{beamcol}{+} \color{SVcol}{D^*}$};
  \fill[SVcol]
    (SV) circle (4pt) node[below right] {SV};
\end{tikzpicture}
}
\legend{Drawing of an event of associated production of $\Upsilon$ and D$^*$. $\Upsilon$ decays into two opposite charge muons, while D$^{*+}$ decays into a pion and D$^0$, which later decays into a kaon and a pion.}
\end{figure}

The D$^*$ reconstruction starts by finding a suitable D$^0$ candidate. For this, two tracks of opposite charge are paired to check whether they form a common vertex using the Kalman Filter method \cite{Fruhwirth:1991pm, Speer:927395}. If the fit is valid and its p-value is greater than 1\%, the D$^0$ candidate is selected. The vertex determination also allows for a better calculation of the kinematic variables of the tracks and it is called secondary vertex (SV).

A primary vertex (PV) is assigned to the D$^0$ candidate as the one closest to the SV in order to calculate the decay length, defined as
\begin{equation}
    dl = \frac{|\Vec{p}_{D^0} \cdot \Delta\Vec{l}|}{|\Vec{p}_{D^0}|},
\end{equation}
where $\Vec{p}_{D^0}$ is the D$^0$ trimomentum and $\Delta\Vec{l}$ is the distance between the PV and SV, the decay length significance is very important because it can be used to filter a lot of background, it is defined as
\begin{equation}
    dl_{sig} = \frac{dl}{dl_{err}},
\end{equation}
where $dl_{err}$ is the uncertainty of $dl$. Finally, another important variable is the cosine of the angle between the PV and SV position vectors (pointing angle)
\begin{equation}
    \cos{\alpha} = \frac{dl}{\Delta\Vec{l}} \; ,
\end{equation}
which hints the alignment between the PV and the SV.

After the D$^0$ candidates is reconstructed, a new track coming from the same PV of the D$^0$ is added to form the D$^*$ candidate and its momentum resolution is improved by adding the PV to the track fit.

The $\Upsilon(nS)$ reconstruction is done in a similar way to the D$^0$. Two muons with opposite charge are selected and its vertex is determined from the fit. If the vertex is valid and has a p-value greater than 1\% we have a dimuon candidate. The invariant mass of the dimuon should be in the range of $8.5 < m_{\mu\mu} < 11.5$ GeV in order to be classified as a $\Upsilon(nS)$ candidate. 

Finally, in order to pair the $\Upsilon(nS)$ and the $D^*$, a vertex fit of the two muons and the slow pion is performed. If valid, the vertex p-value is determined.

In this stage of the reconstruction, very loose cuts are applied to the candidates, those are specified in the Sec. \ref{subsec:preselcuts}.

\section{Event Selection}\label{sec:evtsel}

\subsection{Preselection Cuts} \label{subsec:preselcuts}

Preselection cuts are used to mitigate some of the combinatorial background and save disk space when recording the NTuple. They are very loose and are further improved when the analysis final cuts are applied (Sec. \ref{sec:selcuts}). The preselection cuts both for the $\Upsilon$ and the D$^*$ candidates are summarized in the Tab. \ref{tab:preselectioncuts}.

\begin{table}[!htbp]{15cm}
  \caption{Preselection cuts.}
  \begin{tabular}{ l | l }
    \hline
    \multicolumn{1}{c|}{Variable} & \multicolumn{1}{|c}{Cut} \\ \hline
    \multicolumn{2}{c}{D$^*$ candidate} \\ \hline
    Transverse momentum of $K$ and $\pi$ ($p_T^{K, \pi}$) & $> 0.3$ GeV \\ \hline
    Transverse and longitudinal impact parameter of $K$ and $\pi$ track ($d_{xy}$ and $d_z$) & $< 0.5$ cm \\ \hline
    Transverse and longitudinal impact parameter of $\pi_s$ track ($d_{xy}$ and $d_z$) & $< 2$ cm \\ \hline
    Longitudinal distance between D$^0$ vertex and PV & $< 2$ cm \\ \hline
    Transverse momentum of D$^0$ ($p_T^{D^0}$) & $> 0.9$ GeV \\ \hline
    D$^0$ of D$^{*\pm}$ mass cut ($m_{D^0}$) & $1.5 < m_{D^0} < 2.3$ GeV \\ \hline
    Mass difference between D$^{*\pm}$ and D$^0$ ($\Delta m$) & $< 0.17$ GeV \\ \hline
    $D^0$ candidate vertex probability & $> 0.01$ \\ \hline

    \multicolumn{2}{c}{$\Upsilon$ candidate} \\ \hline
    Pseudorapidity separation between the two $\mu$ ($\Delta\eta_{\mu^+\mu^-}$) & $< 3.0$ \\ \hline
    Transverse and longitudinal impact parameter of the two $\mu$ tracks, & $< 0.5$ cm \\ \hline
    Longitudinal distance between the dimuon candidate vertex and the PV & $< 0.5$ cm \\ \hline
    Dimuon candidate vertex probability & $> 0.01$ \\ \hline
    Dimuon mass range & $8.5 < m_{\mu\mu} < 11.5$ GeV \\ \hline
  \end{tabular}
  \legend{Preselection cuts used to save disk space when saving the NTuples.}
  \label{tab:preselectioncuts}
\end{table}

\textcolor{red}{colocar imagens da pre seleção?}

\subsection{Trigger}

The trigger strategy chosen was to use the HLT Paths that filters dimuons, it was required that the trigger had the maximum possible rapidity coverage, to use for discrimination between DPS and SPS, and lower $p_T$ threshold. The used trigger paths are specified in Tab. \ref{tab:triggers} all of them required two muons with opposite charge, invariant mass between 8.5 and 11.5 GeV, reconstructed pseudorapidity < 2.5 and are unprescaled. In addition to those, there is a cut in transverse momentum different for each trigger path.

In the year 2017, the chosen trigger was not available in the whole data-taking resulting in a lower recorded luminosity (the total was 41.48 fb$^{-1}$ in full 2017 data). This should not pose a problem, since the statistics are large (123.1 fb$^{-1}$ for the three years) and the other triggers did only cover the central area of the detector, which can turn the discrimination between SPS and DPS more complicated. 

\begin{table}[!htbp]{15cm}
  \caption{Triggers used in this study in each year of data taking}
  \begin{tabular}{ c | c | c | c | c }
    Year & Trigger Path & $p_T$ Cut (GeV) & Recorded L ($fb^{-1}$) & L Uncertainty \\  \hline
    2016 & HLT\_Dimuon13\_Upsilon & $>12.9$ & 36.18 & 2.5 \% \\ 
    2017 & HLT\_Dimuon24\_Upsilon\_noCorrL1 & $>13.9$ & 27.12 & 2.3 \% \\
    2018 & HLT\_Dimuon24\_Upsilon\_noCorrL1 & $>13.9$ & 59.82 & 2.5 \% \\
  \end{tabular}
  \legend{Triggers used in this study in each year of data taking. The luminosity uncertainty for each year can be found in \cite{CMS:2017sdi, CMS:2018elu, CMS:2019jhq}}
  \label{tab:triggers}
\end{table}

\subsection{Selection Cuts} \label{sec:selcuts}

The analysis cuts are tighter cuts applied to the data in order to define the fiducial volume and to improve the signal to background ratio. A summary of them are displayed in the Tabs. \ref{tab:fiducialvol} and \ref{tab:selectioncuts}.

\begin{table}[!htbp]{15cm}
  \caption{Kinematic cuts that define the fiducial volume.}
  \begin{tabular}{ l | l }
    \hline
    \multicolumn{1}{c|}{Variable} & \multicolumn{1}{|c}{Cut} \\ \hline
    $\mu$ transverse momentum ($p_T^\mu$) & $> 3$ GeV \\ \hline
    $\mu$ pseudorapidity ($\eta_\mu$) & $|\eta_\mu| < 2.4$ \\ \hline
    $\Upsilon$ transverse momentum ($p_T^\Upsilon$) & $15 < p_T^\Upsilon < 150$ GeV \\ \hline
    $\Upsilon$ rapidity ($y_\Upsilon$) & $|y_\Upsilon| < 2.5$ \\ \hline
    Transverse momentum of $K$ and $\pi$ ($p_T^{K, \pi}$) & $> 1$ GeV \\ \hline
    Transverse momentum of $\pi_s$ ($p_T^{\pi_s}$) & $> 0.3$ GeV \\ \hline
    Transverse momentum of D$^0$ ($p_T^{D^0}$) & $> 3$ GeV\\ \hline
    D$^*$ transverse momentum ($p_T^{D^*}$) & $4 < p_T^{D^*} < 80$ GeV \\ \hline
    D$^*$ rapidity ($y_{D^*}$) & $|y_{D^*}| < 2.5$ \\ \hline
  \end{tabular}
  \legend{Cuts on the kinematic variables of the system that define the fiducial volume of the analysis.}
  \label{tab:fiducialvol}
\end{table}

\begin{table}[!htbp]{15cm}
  \caption{Selection cuts.}
  \begin{tabular}{ l | l }
    \hline
    \multicolumn{1}{c|}{Variable} & \multicolumn{1}{|c}{Cut} \\ \hline
    \multicolumn{2}{c}{D$^*$ candidate} \\ \hline
    Track $\chi^2$ of $K$ and $\pi$ & $< 2.5$ \\ \hline
    Number of valid tracker detector hits for $K$ and $\pi$ & $> 4$ \\ \hline
    Number of valid pixel detector hits for $K$ and $\pi$ & $> 1$ \\ \hline
    transverse impact parameter of $K$ and $\pi$ & 0.5 cm \\ \hline
    longitudinal impact parameter of $K$ and $\pi$ from PV & 0.5/$\sin{\theta}$ \\ \hline
    Track $\chi^2$ of $\pi_s$ & $< 3$ \\ \hline
    Number of valid tracker detector hits for $\pi_s$ & $> 2$\\ \hline
    %$d_{xy}$ of $\pi_s$ from PV & \textcolor{red}{TO ADD} \\ \hline
    %$d_{z}$ of $\pi_s$ from PV & \textcolor{red}{TO ADD} \\ \hline
    D$^0$ mass ($m_{D^0}$) & $|m_{D^0} - 1.864| < 0.028$ GeV \\ \hline
    D$^0$ cosine of the pointing angle ($\cos{\alpha_{D^0}}$) & $> 0.99$ \\ \hline
    D$^0$ decay length significance ($dl_{sig}$) & $> 2.7$ \\ \hline
    
    \multicolumn{2}{c}{$\Upsilon$ candidate} \\ \hline
    $\mu$ soft id flag & soft id = True \\ \hline

    \multicolumn{2}{c}{$\Upsilon$ + D$^*$ candidate} \\ \hline
    $\mu\mu\pi_s$ vertex probability & $> 0.01$ \\ \hline

  \end{tabular}
  \legend{Selection cuts used to improve the signal to background ratio.}
  \label{tab:selectioncuts}
\end{table}

The cuts on Tab. \ref{tab:fiducialvol} were based on studies using MC simulations on the expected acceptance of the detector to each one of the objects. The cuts on the slow pion are more loose to the other tracks, in special, the transverse momentum cut is very important, as it directly correlated to the lower limit of the D$^*$ transverse momentum, since the slow pion phase space is restricted by the low mass difference between D$^0$ and D$^*$. Also, the cut on the D$^0$ observables listed in Tab. \ref{tab:selectioncuts} are very important to improve the D$^*$ signal to background ratio at the expense to reduce greatly the statistics of the sample, so they were very well tunned for this work. 

On Tab. \ref{tab:selectioncuts} the soft id flag refers to the following selections on the muon: 
\begin{itemize}
  \item Muon track in tracker detector matched with at least one segment in the muon detector (in any station) in both X and Y coordinates (< 3$\sigma$).
  \item Number of tracker layers with hits > 5, to guarantee good $p_T$ measurement. 
  \item Number of pixel layers > 0.
  \item Muon track has high-purity flag, rejecting bad quality tracks
  \item Transverse and longitudinal impact parameter cuts, dxy < 0.3 cm and dz < 20 cm w.r.t. the primary vertex.
\end{itemize}

\textcolor{red}{colocar imagens da seleção}

\section{Signal Extraction}

The signal extraction is performed by doing a 2D fit to the dimuon invariant mass ($m_{\mu\mu}$) and D$^*$ $\Delta m$ distributions. To fulfill this a description of the signal and background for each distribution is created and a product of both is taken as the 2D model. Each of the distribution models contains components for signal and background and a non-extended composite PDF is constructed. The fit is performed using the RooFit package.

\subsection{\texorpdfstring{$\Upsilon$}{Y} Model}

The fit model has three signal components, one for each of the observed $\Upsilon(nS)$ states. All of them are modeled using the Crystal Ball distribution, defined as
\begin{equation}
  CB(x;\alpha,n,\bar x,\sigma) = N \cdot 
  \begin{cases} 
    \exp(- \frac{(x - \bar x)^2}{2 \sigma^2}), & \mbox{for }\frac{x - \bar x}{\sigma} > -\alpha \\
    A \cdot (B - \frac{x - \bar x}{\sigma})^{-n}, & \mbox{for }\frac{x - \bar x}{\sigma} \leqslant -\alpha 
  \end{cases}
\end{equation}
where
\begin{equation}
\begin{split}
  A = & \left(\frac{n}{\left| \alpha \right|}\right)^n \cdot \exp\left(- \frac {\left| \alpha \right|^2}{2}\right), \\
  B = & \frac{n}{\left| \alpha \right|}  - \left| \alpha \right|, \\
  N = & \frac{1}{\sigma (C + D)}, \\
  C = & \frac{n}{\left| \alpha \right|} \cdot \frac{1}{n-1} \cdot \exp\left(- \frac {\left| \alpha \right|^2}{2}\right), \\
  D = & \sqrt{\frac{\pi}{2}} \left(1 + \operatorname{erf}\left(\frac{\left| \alpha \right|}{\sqrt 2}\right)\right). \\
\end{split}
\end{equation}
With N as the normalization factor, $\alpha$, $n$, $\bar x$ and $\sigma$ are the fit parameters and erf is the error function. This function behaves like a Gaussian for $(x-\bar x)/\sigma$ greater than $-\alpha$ and presents a tail for values less than equals $-\alpha$, being able to model the effects of FSR to the dimuon invariant mass distribution.

In order to reduce the amount of free parameters of the fit, the following constraints are imposed to the CB parameters:
\begin{itemize}
  \item The mean of all the CBs are fixed to the PDG mass value of its $\Upsilon$ state times a mass scale (to take into account the uncertainties on muon momentum scale calibration). $\bar x_{\Upsilon(nS)} = m_{scale}\cdot m_{\Upsilon(nS)}$;
  \item The sigma of the CBs used to model 2S and 3S states are set to be proportional to their mass ratio with respect to the 1S state. Therefore, only one sigma is taken as free parameter. $\sigma_{\Upsilon(2S)} = \sigma_{1S}\times m_{\Upsilon(2S)}/m_{\Upsilon(1S)}$ and $\sigma_{\Upsilon(3S)} = \sigma_{1S}\times m_{\Upsilon(3S)}/m_{\Upsilon(1S)}$;
  \item The tail parameters ($\alpha$ and $n$) are set as the same for the three CBs;
  \item The parameter $n$ is set to be constant and its value is chosen from a previous 1D fit on the dimuon mass distribution.
\end{itemize}

Finally, the fit model for $\Upsilon(nS)$ signal is
\begin{equation} \label{eq:upsilon_sig}
  S_{\Upsilon}(m_{\mu\mu}) = f_{\Upsilon(1S)}\cdot CB_{\Upsilon(1S)} (m_{\mu\mu}) + f_{\Upsilon(2S)}\cdot CB_{\Upsilon(2S)} (m_{\mu\mu}) + (1-f_{\Upsilon(1S)}-f_{\Upsilon(2S)}) \cdot CB_{\Upsilon(3S)} (m_{\mu\mu})
\end{equation}

For the background, Chebyshev polynomials of the first kind are used up to second order. They are defined as:
\begin{equation} \label{eq:upsilon_bkg}
\begin{split}
  T_0(x) = & 1 \\
  T_1(x) = & x \\
  T_{n+1}(x) = & 2xT_n(x) - T_{n-1}(x).
\end{split}
\end{equation}
The background component is written as
\begin{equation}
  B_{\Upsilon}(m_{\mu\mu}) = N[1 + b_1 T_1(m_{\mu\mu}) + b_2 T_2(m_{\mu\mu})],
\end{equation}
where N is the normalization constant and $b_1$ and $b_2$ as free parameters for the fit.

The final fit model is a sum of the four components with recursive coefficients:

\subsection{\texorpdfstring{D$^{*}$}{D*} Model}

For D$^*$ signal component, it is used a Johnson's distribution
\begin{equation}\label{eq:dstar_sig}
  S_{D^*}(\Delta m) = \frac{\delta}{\lambda\sqrt{2\pi}} \frac{1}{\sqrt{1 + \left(\frac{\Delta m-\mu}{\lambda}\right)^2}}\exp{\left[-\frac{1}{2}\left(\gamma+\delta\sinh^{-1}\left(\frac{\Delta m-\mu}{\lambda}\right)\right)^2\right]},
\end{equation}
where $\delta$, $\lambda$, $\gamma$ and $\mu$ are the free parameters. This function is often used to describe the mass difference of charm decays and fits well the $\Delta$m distribution.

For the D$^*$ background, a threshold function \cite{ZEUS:2013fws} given by:
\begin{equation} \label{eq:dstar_bkg}
  B_{D^*}(\Delta m) = A \cdot (\Delta m - m_\pi)^B \cdot \exp[-C\cdot \Delta m]
\end{equation}
is used, where $m_\pi$ is the pion mass and $A$, $B$ and $C$ are free parameters.

\subsection{\texorpdfstring{$\Upsilon$ + D$^{*}$}{Y+D*} 2D Model}

Finally, the construction of the 2D model takes into account the 4 distributions from Eqs. \ref{eq:upsilon_sig}, \ref{eq:upsilon_bkg}, \ref{eq:dstar_sig} and \ref{eq:dstar_bkg}, creating the final four component distribution:
\begin{equation}
\begin{split}
  M_{\Upsilon D^*}(m_{\mu\mu}, \Delta m) & = f_s \cdot S_\Upsilon(m_{\mu\mu}) \cdot S_{D^*}(\Delta m) \\
  & + f_{b1} \cdot S_\Upsilon(m_{\mu\mu}) \cdot B_{D^*}(\Delta m) \\
  & + f_{b1} \cdot B_\Upsilon(m_{\mu\mu}) \cdot S_{D^*}(\Delta m) \\ 
  & + (1-f_{b1}-f_{b2}) \cdot B_\Upsilon(m_{\mu\mu}) \cdot B_{D^*}(\Delta m)
\end{split}
\end{equation}

The firs component, composed by both signal models, is used to estimate the associated $\Upsilon$ and $D^*$ yield.

\textcolor{red}{colocar imagens do fit}

\section{Acceptance and Efficiency}

The acceptance and efficiency of the channel is determined from the signal MC. The strategy for computing the acceptance and efficiency is to factorize it into the components,
\begin{equation}
  (Acc\cdot\epsilon) = (Acc\cdot\epsilon_{precuts})^\Upsilon \cdot 
  (Acc\cdot\epsilon_{precuts})^{D^*} \cdot \epsilon_{cuts}^\Upsilon \cdot 
  \epsilon_{cuts}^{D^*} \cdot \epsilon_{HLT} \cdot \epsilon_{association},
\end{equation}
where the $Acc\cdot\epsilon_{precuts}$ is the acceptance coupled to the precuts efficiency, $\epsilon_{cuts}$ is the analysis cuts efficiency, $\epsilon_{HLT}$ is the trigger efficiency and the $\epsilon_{association}$ is the efficiency related to the $\Upsilon-D^*$ association criteria. Sec. \ref{sec:evtsel} gives details of the precuts and cuts applied.

\subsection{Acceptance}

The acceptance is calculated taking into account the precuts (Tab. \ref{tab:preselectioncuts}) and the cuts that define the fiducial volume of the sample on Tab. \ref{tab:fiducialvol}. The acceptance coupled to the precuts is calculated by
\begin{equation}
    (Acc \cdot \epsilon_{precuts})^P = \frac{N_{reco}^P}{N_{gen}^P},
\end{equation}
where the superscript P refers either to $\Upsilon$ or $D^*$, $N_gen$ is the number of generated particles within the fiducial volume, and $N_reco$ is the number of reconstructed particles within the same fiducial volume, passing the precuts and satisfying the respective matching criteria for that kind of particle:
\begin{itemize}
  \item For $\Upsilon$: the two muons used for its reconstruction must be matched to the muons from the decay of the generated Y within a cone of $\Delta R = \sqrt{\Delta \eta^2 + \Delta \phi^2} < 0.03$.
  \item For $D^*$: The reconstructed $D^*$ matches the generated $D^*$ with the requirements:
  $$\frac{|p_T^{reco} - p_T^{gen}|}{p_T^{gen}} < 0.2, |\eta_{gen} - \eta_{reco}|
  < 0.3, remainder(|\phi_{gen} - \phi_{reco}|, 2\pi) < 0.3.$$
\end{itemize}

\textcolor{red}{TO ADD FIGURES of the maps}

\subsection{Cuts Efficiency}

The cuts considered for this efficiency component are the ones stated in the Tab. \ref{tab:selectioncuts} with exception to the cut on the $\mu\mu\pi_s$ vertex probability cut, which is treated separately. The denominator is the number of reconstructed events that passed the precuts criteria and the numerator is the number of events that passed the cuts and the precuts:
\begin{equation}
  \epsilon_{cuts}^P = \frac{N_{reco\&cuts}^P}{N_{reco}^P} (P = \Upsilon, D^*).
\end{equation}

\textcolor{red}{TO ADD FIGURES on the maps}

\subsection{Trigger Efficiency}

The triggers used depend only on the dimuons, therefore, their efficiency is only evaluated from the 
$\Upsilon$ candidates. The denominator is the number of events that passed both the precuts and cuts criteria and the numerator the number of events passing the trigger, cuts and precuts:
\begin{equation}
  \epsilon_{HLT} = \frac{N_{reco\&cuts\&trigger}^\Upsilon}{N_{reco\&cuts}^\Upsilon}
\end{equation}

\textcolor{red}{TO ADD FIGURES on the maps}

\subsection{Association efficiency}

The last efficiency is related to the association between $\Upsilon$ and $D^*$. It is given by
\begin{equation}
  \epsilon_{association} = 
  \frac{N_{reco\&cuts\&trigger\&association}^{\Upsilon D^*}}{N_{reco\&cuts\&trigger}^{\Upsilon D^*}},
\end{equation}
where the denominator is the number events that passed all the previous selections and the numerator is the number of events after the addition of the association cut, which is the $\mu\mu\pi_s$ vertex probability $> 0.01$.

\textcolor{red}{TO ADD FIGURES on the maps}

The shown efficiency maps are used to calculate efficiency that will be used in the cross section determination.

\section{Cross Section Measurement}


