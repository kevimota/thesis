\pretextualchapter{Abstract}
\reference 

This work presents the analysis of the associated production of $\Upsilon$ and D$^{*\pm}$ using pp collision data collected by CMS, at $\sqrt{s} = 13$ TeV, with integrated luminosity of 123 fb$^{-1}$. The associated production of $\Upsilon$ and open charm hadrons are considered a golden channel to study the double parton scattering (DPS), as the single parton scattering contribution (SPS) is expected to be negligible.

The reconstruction of the two mesons is done through $\Upsilon(nS) \rightarrow \mu^+\mu^-$ and D$^{*\pm} \rightarrow K^\mp\pi^\pm \pi^\pm$ decay channels. The fiducial cross section is calculated and used to compute the effective cross section by assuming that the production is exclusively done via DPS.

Apart from the analysis, this document also discuss the work done for the CMS RPC project on the maintanance and commissioning, in preparation for the new data taking period (Run 3), and the R\&D of the new iRPC detectors expected to be installed in CMS during the Phase 2 upgrade.
~\\

\printkeys 

\pretextualchapter{Resumo}
\referencia 

Esse trabalho apresenta a análise da produção associada de $\Upsilon$ e D$^{*\pm}$ usando dados de colisões pp coletadas pelo CMS, em $\sqrt{s} = 13$ TeV, com luminosidade integrada de 123 fb$^{-1}$. A produção associada de $\Upsilon$ e hádrons \textit{open charm} é considerada muito interessante para estudar o espalhamento duplo de pártons (da sigla em inglês, DPS),  uma vez que é esperado que contribuição do espalhamento simples de pártons (da sigla em inglês, SPS) é negligível.

A reconstrução dos dois mésons é feita pelos canais de decaimento $\Upsilon(nS) \rightarrow \mu^+\mu^-$ e D$^{*\pm} \rightarrow K^\mp\pi^\pm \pi^\pm$. A seção de choque fiducial é computada e usada para calcular a seção de choque efetiva assumindo que a produção é feita exclusivamente via DPS.

Além da análise, esse documento também discute os trabalhos desenvolvidos na manutenção e comissionamento dos detectores do projeto CMS RPC, em preparação para o novo período de tomada de dados (Run 3), e da P\&D dos novos detectores iRPC que devem ser instalados no CMS durante o upgrade fase 2.
~\\

\imprimirchaves 
